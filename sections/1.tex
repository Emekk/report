\section{Overview}
\label{sec:overview}

\paragraph{} With its \textit{virtual (network) structure}, Nike has a global network of over 500 supplier factories from 37 countries. Managing such a large network of suppliers is a challenge, especially when it comes to ensuring that the suppliers comply with the company's standards. Below are the pros and cons of this structure.

\begin{table}[H]
    \centering
    \begin{tabular}{|l|l|}
        \hline
        Pros & Cons \\
        \hline
        - Increased productivity & - Lack of control over suppliers \\
        - Cost efficiency & - Problems with partners may occur \\
        \hline
    \end{tabular}
    \caption{Pros and cons of Nike's virtual (network) structure}
    \label{tab:virtual-structure}
\end{table}

\paragraph{} Nike has been criticized for its poor working conditions in its supplier factories, and the company has taken several steps to improve the situation. In this report, we will discuss the ethical issues that Nike faced, and the actions that the company took to address them. Finally, we will discuss the current challenges that Nike faces, and provide recommendations for the company to overcome them.