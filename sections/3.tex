\section{Supply Chain Improvements}

\subsection{Country Risk Index and Manufacturing Index}

\paragraph{} Nike created the Country Risk Index which is a measure of the risk of doing business in a country. It is calculated based on factors such as labor conditions, environmental regulations, economy and corruption. Downside of this index is its subjectivity.

\paragraph{} Nike also created the Manufacturing Index which asesses the suppliers in four areas: cost, quality, delivery, sustainability. As a positive side, this index uses \textit{holistic approach}; however, as a negative side, its over-simplification may lead to missing nuances.

\paragraph{} While these indices may seem useful, they are shallow ways of assessing the suppliers, which may misdirect the company to miss opportunities or to make wrong decisions. A more detailed and holistic approach, such as system dynamics, would be more useful. 

\subsection{Reorganizing Global Sourcing \& Manufacturing}

\paragraph{} Nike sought to hire a professional with experience in labor rights and environmental issues. The reason for this was that the current employees had been working in the company for a long time and might be under the influence of \textit{operational blindness}.
