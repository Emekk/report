\section{Ethical Issues Nike Faced}

\subsection{Labor Exploitation}

\paragraph{} Nike faced many ethical issues, such as substandard housing, low wages, ocuupational health and safety problems, child labor and even human trafficking. Initially, Nike denied the responsibility with arguments such as, \enquote{\textit{It is not within Nike’s scope to investigate labor violations.}}, \enquote{\textit{It is subcontractor's fault}} and \enquote{\textit{We are contributing to the economies of undeveloped countries and woman workforce.}}. 

\subsection{First Actions}

\paragraph{} After defending itself, Nike also took some actions to address the issues. The first one was to create a \textbf{Code of Conduct} which defines and regulates the working conditions of the supplier factories. Second one was to establish the \textbf{Corporate Social Responsibility} (CSR) department. Pros and cons of these actions are discussed below.

\begin{table}[H]
    \centering
    \begin{tabular}{|l|l|}
        \hline
        Pros & Cons \\
        \hline
        - Initiating a legal base & - Easy to evade by suppliers \\
        - Transparency & - Hard to supervise and maintain \\
        - Systematic approach of corporate responsibility & \\
        \hline
    \end{tabular}
    \caption{Pros and cons of the first actions taken by Nike}
    \label{tab:first-actions}
\end{table}

\paragraph{} These actions were, however, not enough to solve the problem. The two main reasons for this were the lack of supplier collaboration and the cultural and legal differences between the countries.
