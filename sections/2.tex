\section{Ethical Issues Nike Faced}

\subsection{Labor Exploitation}

\paragraph{} As mentioned in \textsl{\nameref{sec:overview}}, managing a large network of suppliers is a challenge, with a lack of control over suppliers being one of the main problems. This lack of control can lead to labor exploitation, which is what happened in Nike's case. The company faced many ethical issues, such as substandard housing, low wages, ocuupational health and safety problems, child labor and even human trafficking.

\paragraph{} Initially, Nike denied the responsibility with arguments such as, \enquote{\textit{It is not within Nike’s scope to investigate labor violations.}}, \enquote{\textit{It is subcontractor's fault}} and \enquote{\textit{We are contributing to the economies of undeveloped countries and woman workforce.}}. 

\subsection{First Actions}

\paragraph{} After defending itself, Nike also took some actions to address the issues. The first one was to create a \textbf{Code of Conduct} which defines and regulates the working conditions of the supplier factories. Second one was to establish the \textbf{Corporate Social Responsibility} (CSR) department. Pros and cons of these actions are discussed below.

\begin{table}[H]
    \centering
    \begin{tabular}{|l|l|}
        \hline
        Pros & Cons \\
        \hline
        - Initiating a legal base & - Easy to evade by suppliers \\
        - Transparency & - Hard to supervise and maintain \\
        - Systematic approach of corporate responsibility & \\
        \hline
    \end{tabular}
    \caption{Pros and cons of first actions taken by Nike}
    \label{tab:first-actions}
\end{table}

\paragraph{} These actions were, however, not enough to solve the problem. The two main reasons for this are:
\begin{itemize}[label=\ding{70}]
    \item \textbf{Lack of supplier collaboration:} Inadequate collaboration and communication with suppliers led to ethical noncompliance.
    \item \textbf{Cultural and legal differences:} Operating in different countries means dealing with diverse cultural norms and legal frameworks.

\end{itemize}